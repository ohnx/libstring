
\begin{DoxyCode}
                 \_ \_ \_         \_        \_             
                | (\_) |       | |      (\_)            
                | |\_| |\_\_  \_\_\_| |\_ \_ \_\_ \_ \_ \_\_   \_\_ \_ 
                | | | \textcolor{stringliteral}{'\_ \(\backslash\)/ \_\_| \_\_| '}\_\_| | \textcolor{stringliteral}{'\_ \(\backslash\) / \_` |}
\textcolor{stringliteral}{                | | | |\_) \(\backslash\)\_\_ \(\backslash\) |\_| |  | | | | | (\_| |}
\textcolor{stringliteral}{                |\_|\_|\_.\_\_/|\_\_\_/\(\backslash\)\_\_|\_|  |\_|\_| |\_|\(\backslash\)\_\_, |}
\textcolor{stringliteral}{                                                 \_\_/ |}
\textcolor{stringliteral}{                           by ohnx              |\_\_\_/ }
\textcolor{stringliteral}{}
\textcolor{stringliteral}{            a simple flexible string manipulation library}
\textcolor{stringliteral}{}
\textcolor{stringliteral}{ABOUT}
\textcolor{stringliteral}{    libstring is a simple flexible string manipulation library,}
\textcolor{stringliteral}{    written in C89. All library functions are able to take both}
\textcolor{stringliteral}{    C-style char arrays and libstring-style strings as input}
\textcolor{stringliteral}{    (output from is always libstring-style strings).}
\textcolor{stringliteral}{}
\textcolor{stringliteral}{    libstring keeps track of memory when appending strings -}
\textcolor{stringliteral}{    so you don'}t have to! The only time when you need to manual
    memory manage is when you\textcolor{stringliteral}{'re done with a string.}
\textcolor{stringliteral}{}
\textcolor{stringliteral}{GOALS}
\textcolor{stringliteral}{    libstring tries to provide a simpler way to deal with strings}
\textcolor{stringliteral}{    in C, so that a programmer can focus on other aspects of a}
\textcolor{stringliteral}{    project. It aims to be lightweight (it only has ~5 real}
\textcolor{stringliteral}{    functions) yet powerful. }
\textcolor{stringliteral}{}
\textcolor{stringliteral}{LIBSTRING-STYLE STRINGS}
\textcolor{stringliteral}{    libstring-style strings can be treated as normal C-style}
\textcolor{stringliteral}{    strings. (ie, all C str* functions are compatible with}
\textcolor{stringliteral}{    libstring-style strings, provided they do not increase the}
\textcolor{stringliteral}{    length of the string)}
\textcolor{stringliteral}{}
\textcolor{stringliteral}{    How does this work? Well, libstring-style strings are actually}
\textcolor{stringliteral}{    char arrays. However, they have an extra header in front.}
\textcolor{stringliteral}{    When you call a libstring function, it returns only a pointer}
\textcolor{stringliteral}{    to the char array afterwards - not the header. This means that}
\textcolor{stringliteral}{    you can call functions that expect a char array by passing}
\textcolor{stringliteral}{    them any libstring-returned string.}
\textcolor{stringliteral}{}
\textcolor{stringliteral}{    If you are concerned about reads before the start of a string,}
\textcolor{stringliteral}{    string\_mknew() will always assume the string passed in to it}
\textcolor{stringliteral}{    is a non library-compatible string, and return a pointer to a}
\textcolor{stringliteral}{    newly-allocated library-comptabile string.}
\textcolor{stringliteral}{}
\textcolor{stringliteral}{    Internally, these strings are represented by a struct of type}
\textcolor{stringliteral}{    `string\_real`. See the documentation generated by doxygen for}
\textcolor{stringliteral}{    more information on how libstring works behind the scenes.}
\textcolor{stringliteral}{}
\textcolor{stringliteral}{BUILDING}
\textcolor{stringliteral}{    Run `make` on systems with make and gcc. This will generate a}
\textcolor{stringliteral}{    file `libstring.a`. To test libstring, you can also run}
\textcolor{stringliteral}{    `make test` to compile and run the tests.}
\textcolor{stringliteral}{}
\textcolor{stringliteral}{DOCUMENTATION}
\textcolor{stringliteral}{    Run `make docs` on systems with make and doxygen. This will}
\textcolor{stringliteral}{    generate the HTML and LaTeX documentation in the folder `docs/`}
\textcolor{stringliteral}{}
\textcolor{stringliteral}{EXAMPLES}
\textcolor{stringliteral}{    See the link: https://github.com/ohnx/libstring/wiki/EXAMPLES}
\textcolor{stringliteral}{}
\textcolor{stringliteral}{LICENSE}
\textcolor{stringliteral}{    MIT-licensed, see the LICENSE file.}
\textcolor{stringliteral}{}
\textcolor{stringliteral}{A NOTE ON PRINTF}
\textcolor{stringliteral}{    libstring'}s printf implementation relies on va\_copy, which
    is not part of c89. Instead, it is available as an extension.
    To compile libstring with printf support, run `make printf`.
    Keep in mind that the printf code compiles as c99.
\end{DoxyCode}
 